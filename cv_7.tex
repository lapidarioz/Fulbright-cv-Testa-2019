%%%%%%%%%%%%%%%%%%%%%%%%%%%%%%%%%%%%%%%%%
% "ModernCV" CV and Cover Letter
% LaTeX Template
% Version 1.11 (19/6/14)
%
% This template has been downloaded from:
% http://www.LaTeXTemplates.com
%
% Original author:
% Xavier Danaux (xdanaux@gmail.com)
%
% License:
% CC BY-NC-SA 3.0 (http://creativecommons.org/licenses/by-nc-sa/3.0/)
%
% Important note:
% This template requires the moderncv.cls and .sty files to be in the same 
% directory as this .tex file. These files provide the resume style and themes 
% used for structuring the document.
%
%%%%%%%%%%%%%%%%%%%%%%%%%%%%%%%%%%%%%%%%%

%----------------------------------------------------------------------------------------
%	PACKAGES AND OTHER DOCUMENT CONFIGURATIONS
%----------------------------------------------------------------------------------------

\documentclass[11pt,a4paper,sans]{moderncv} % Font sizes: 10, 11, or 12; paper sizes: a4paper, letterpaper, a5paper, legalpaper, executivepaper or landscape; font families: sans or roman
\usepackage[utf8]{inputenc}

\moderncvstyle{classic} % CV theme - options include: 'casual' (default), 'classic', 'oldstyle' and 'banking'
\moderncvcolor{blue} % CV color - options include: 'blue' (default), 'orange', 'green', 'red', 'purple', 'grey' and 'black'

\usepackage{lipsum} % Used for inserting dummy 'Lorem ipsum' text into the template

\usepackage[scale=.85]{geometry} % Reduce document margins
%\setlength{\hintscolumnwidth}{3cm} % Uncomment to change the width of the dates column
%\setlength{\makecvtitlenamewidth}{10cm} % For the 'classic' style, uncomment to adjust the width of the space allocated to your name

%----------------------------------------------------------------------------------------
%	NAME AND CONTACT INFORMATION SECTION
%----------------------------------------------------------------------------------------

\firstname{Rafael Luiz} % Your first name
\familyname{Testa} % Your last name

% All information in this block is optional, comment out any lines you don't need
\title{Curriculum Vitae}
\address{University of São Paulo}{Brazil}
\email{rafael.testa@usp.br}
\homepage{http://lapis.each.usp.br} {http://lapis.each.usp.br} % The first argument is the url for the clickable link, the second argument is the url displayed in the template - this allows special characters to be displayed such as the tilde in this example
% \extrainfo{DOB: Month Day, Year}
\photo[70pt][0.4pt]{pictures/foto} % The first bracket is the picture height, the second is the thickness of the frame around the picture (0pt for no frame)
%\quote{"A witty and playful quotation" - John Smith}
%----------------------------------------------------------------------------------------

\begin{document}

\makecvtitle % Print the CV title
%----------------------------------------------------------------------------------------
%	EDUCATION SECTION
%----------------------------------------------------------------------------------------
%Full Professor at the University of São Paulo. Bachelor in Computer Science (Universidade Estadual Paulista Julio de Mesquita), Master in Electrical Engineering (University of São Paulo) and PhD in Science (Computational Physics) (University of São Paulo). Her research interests include Virtual Reality, Image Processing, Content-based multimedia data retrieval. Several of her works are applied in healthcare, establishing interfaces with the of Biomedical Engineering area.

\section{Education}

\cventry{2019--Present}{Ph.D. in Information Systems}{University of São Paulo}{}{}{}

\cventry{2016--2018}{Master of Science in Information Systems}{University of São Paulo}{}{}{}

\cventry{1997--2001}{Bachelor in Information Systems}{University of São Paulo}{}{}{}

%----------------------------------------------------------------------------------------
%	AWARDS SECTION
%----------------------------------------------------------------------------------------
\section{Positions}

\cventry{2019--Present}{Ph.D. Fellow}{Coordination for the Improvement of Higher Education Personnel (CAPES)}{}{}{Title: Facial expression synthesis with different emotion intensity levels}

\cventry{2017--2018}{Master Fellow}{Coordination for the Improvement of Higher Education Personnel (CAPES)}{}{}{Title: Facial expression synthesis based on similar faces}

\cventry{2015--2016}{Web Developer and DevOps}{JP7}{}{}{}

\cventry{2013--2014}{Fellow of Undergraduate Research Mentorsip Program}{National Council for Scientific and Technological Development (CNPq)}{}{}{Title: Generating Facial Emotions using anthropometry and grammars for Diagnosis and Training}

\cventry{2012--2013}{Fellow of Undergraduate Research Mentorsip Program}{National Council for Scientific and Technological Development (CNPq)}{}{}{Title:Interaction data recording using Aspect Oriented Programming in a system for assessing knowledge acquisition in three-dimensional Virtual Learning Environments}


%----------------------------------------------------------------------------------------
%	Teaching
%----------------------------------------------------------------------------------------

\section{Teaching}

\cventry{2017}{Teaching Assistant}{Database Management System Laboratory}{University of São Paulo}{}{}

%----------------------------------------------------------------------------------------
%	Masters SECTION
%----------------------------------------------------------------------------------------

%\section{Masters Thesis}

%\cvitem{Title}{\emph{Technologies and characterization of ferroelectric polymers for biomedical sensors}}
%\cvitem{Supervisors}{Professor Antonino Fiorillo}
%\cvitem{Description}{This thesis is based on the implementation of a temperature sensor.}

%----------------------------------------------------------------------------------------
%	WORK EXPERIENCE SECTION
%----------------------------------------------------------------------------------------

\section{Publications}
\begin{enumerate}
    \item TESTA, RAFAEL LUIZ; CORRÊA, CLÉBER GIMENEZ ; MACHADO-LIMA, ARIANE ; NUNES, FÁTIMA L. S. . Synthesis of Facial Expressions in Photographs. ACM COMPUTING SURVEYS, v. 51, p. 1-35, 2019.
    \item  TESTA, RAFAEL LUIZ; MACHADO LIMA, ARIANE ; NUNES, FATIMA DE LOURDES DOS SANTOS . Factors Influencing the Perception of Realism in Synthetic Facial Expressions. In: 2018 31st SIBGRAPI Conference on Graphics, Patterns and Images (SIBGRAPI), 2018, Paraná. 2018 31st SIBGRAPI Conference on Graphics, Patterns and Images (SIBGRAPI), 2018. p. 297.
    \item  TESTA, RAFAEL LUIZ; MUNIZ, ANTONIO HENRIQUE NUNES ; CARPIO, LISETH URPY SEGUNDO ; DIAS, RODRIGO DA SILVA ; ROCCA, CRISTIANA CASTANHO DE ALMEIDA ; LIMA, ARIANE MACHADO ; MARQUES, FATIMA DE LOURDES DOS SANTOS NUNES . Generating Facial Emotions for Diagnosis and Training. In: 2015 IEEE 28th International Symposium on ComputerBased Medical Systems (CBMS), 2015, Sao Carlos. 2015 IEEE 28th International Symposium on Computer-Based Medical Systems. v. 1. p. 304-309.
    \item  NUNES, E. P. S. ; TESTA, R. L. ; NUNES, F. L. S. ; ROQUE, L. G. . Um Estudo Experimental sobre a Captura Automática dos Dados de Interação em Ambientes Virtuais Tridimensionais. In: XIII Simpósio Brasileiro Sobre Fatores Humanos em Sistemas Computacionais,. In: XIII Simpósio Brasileiro Sobre Fatores Humanos em Sistemas Computacionais, 2014, Foz do Iguaçu-PR. Proceedings of the 13th Brazilian Symposium on Human Factors in Computing Systems, 2014. p. 313-322.
    \item  TESTA, R. L.; NUNES, E. P. S. ; NUNES, F. L. S. . Uma Revisão Sistemática sobre o Registro das Interações do Usuário em Ambientes Virtuais de Aprendizagem Tridimensionais. In: X Workshop de Realidade Virtual e Aumentada - WRVA'2013, 2013, Jataí/GO. Realidade Virtual e Aumentada: Anais do WRVA 2013, 2013.
    \item  NUNES, E. P. S. ; TESTA, R. L. ; NUNES, F. L. S. . Sistema de Coleta de Dados para Avaliação da Aprendizagem em Ambientes Virtuais de Aprendizagem Tridimensionais. In: VI Workshop sobre Avaliação e Acompanhamento da Aprendizagem em Ambientes Virtuais - WAVALIA, 2013, CA,pinas. Anais do II Congresso Brasileiro de Informática na Educação, 2013.
    \item  TESTA, R. L.; MUNIZ, A. H. N. ; MACHADO-LIMA, A ; NUNES, F. L. S. . Geração de expressões faciais a partir de antropometria e gramáticas para aplicação em jogos psiquiátricos. In: 22º Simpósio Internacional de Iniciação Científica da USP - SIICUSP, 2014, São Paulo. Anais do 22º Simpósio Internacional de Iniciação Científica da USP - SIICUSP, 2014.
    \item  MUNIZ, A. H. N. ; TESTA, R. L. ; NUNES, F. L. S. ; MACHADO-LIMA, A . Definição e implementação de gramática para geração de expressões faciais em jogos psiquiátricos. In: 22º Simpósio Internacional de Iniciação Científica da USP - SIICUSP, 2014, São Paulo. Anais do 22º Simpósio Internacional de Iniciação Científica da USP - SIICUSP, 2014.
    \item  TESTA, R. L.; NUNES, E. P. S. ; NUNES, F. L. S. . Monitoramento das Interações dos Usuários em Ambientes Virtuais de Aprendizagem Tridimensionais.. In: 21º Simpósio Internacional de Iniciação Científica da USP - SIICUSP, 2013, São Carlos. Anais do 21º Simpósio Internacional de Iniciação Científica da USP - SIICUSP, 2013.
\end{enumerate}
\section{Volunteer Work}

\cventry{2019 --
Present}{Math Teacher}{Preparatory course for college entrance examination}{Emancipa Popular Course}{Ordalina Cândido unit}{The course targets low-income, underperforming students.}

%----------------------------------------------------------------------------------------
%	COMMUNICATION SKILLS SECTION
%----------------------------------------------------------------------------------------

\section{Communication Skills}

\cvitem{2018}{Oral Presentation at the Conference on Graphics, Patterns and Images (SIBGRAPI)}
\cvitem{2018}{Oral Presentation at the 5th PPgSI’s Dissertations Workshop}
\cvitem{2017}{Oral Presentation at the Semana da ciência}
\cvitem{2017}{Oral Presentation at the 4th PPgSI’s Dissertations Workshop}
\cvitem{2017}{Poster presentation at the Encontro Paulista de Pós-Graduandos em Computação (EPPC)}
\cvitem{2014}{Poster presentation at the 22º Simpósio Internacional de Iniciação Científica da USP}
\cvitem{2013}{Poster presentation at the 21º Simpósio Internacional de Iniciação Científica da USP}


\end{document}



